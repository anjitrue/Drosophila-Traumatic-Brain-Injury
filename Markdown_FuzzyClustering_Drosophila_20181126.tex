\documentclass[]{article}
\usepackage{lmodern}
\usepackage{amssymb,amsmath}
\usepackage{ifxetex,ifluatex}
\usepackage{fixltx2e} % provides \textsubscript
\ifnum 0\ifxetex 1\fi\ifluatex 1\fi=0 % if pdftex
  \usepackage[T1]{fontenc}
  \usepackage[utf8]{inputenc}
\else % if luatex or xelatex
  \ifxetex
    \usepackage{mathspec}
  \else
    \usepackage{fontspec}
  \fi
  \defaultfontfeatures{Ligatures=TeX,Scale=MatchLowercase}
\fi
% use upquote if available, for straight quotes in verbatim environments
\IfFileExists{upquote.sty}{\usepackage{upquote}}{}
% use microtype if available
\IfFileExists{microtype.sty}{%
\usepackage{microtype}
\UseMicrotypeSet[protrusion]{basicmath} % disable protrusion for tt fonts
}{}
\usepackage[margin=1in]{geometry}
\usepackage{hyperref}
\hypersetup{unicode=true,
            pdftitle={Drosophila Traumatic Brain Injury},
            pdfauthor={Anji Trujillo - Professor Coon},
            pdfborder={0 0 0},
            breaklinks=true}
\urlstyle{same}  % don't use monospace font for urls
\usepackage{color}
\usepackage{fancyvrb}
\newcommand{\VerbBar}{|}
\newcommand{\VERB}{\Verb[commandchars=\\\{\}]}
\DefineVerbatimEnvironment{Highlighting}{Verbatim}{commandchars=\\\{\}}
% Add ',fontsize=\small' for more characters per line
\usepackage{framed}
\definecolor{shadecolor}{RGB}{248,248,248}
\newenvironment{Shaded}{\begin{snugshade}}{\end{snugshade}}
\newcommand{\KeywordTok}[1]{\textcolor[rgb]{0.13,0.29,0.53}{\textbf{#1}}}
\newcommand{\DataTypeTok}[1]{\textcolor[rgb]{0.13,0.29,0.53}{#1}}
\newcommand{\DecValTok}[1]{\textcolor[rgb]{0.00,0.00,0.81}{#1}}
\newcommand{\BaseNTok}[1]{\textcolor[rgb]{0.00,0.00,0.81}{#1}}
\newcommand{\FloatTok}[1]{\textcolor[rgb]{0.00,0.00,0.81}{#1}}
\newcommand{\ConstantTok}[1]{\textcolor[rgb]{0.00,0.00,0.00}{#1}}
\newcommand{\CharTok}[1]{\textcolor[rgb]{0.31,0.60,0.02}{#1}}
\newcommand{\SpecialCharTok}[1]{\textcolor[rgb]{0.00,0.00,0.00}{#1}}
\newcommand{\StringTok}[1]{\textcolor[rgb]{0.31,0.60,0.02}{#1}}
\newcommand{\VerbatimStringTok}[1]{\textcolor[rgb]{0.31,0.60,0.02}{#1}}
\newcommand{\SpecialStringTok}[1]{\textcolor[rgb]{0.31,0.60,0.02}{#1}}
\newcommand{\ImportTok}[1]{#1}
\newcommand{\CommentTok}[1]{\textcolor[rgb]{0.56,0.35,0.01}{\textit{#1}}}
\newcommand{\DocumentationTok}[1]{\textcolor[rgb]{0.56,0.35,0.01}{\textbf{\textit{#1}}}}
\newcommand{\AnnotationTok}[1]{\textcolor[rgb]{0.56,0.35,0.01}{\textbf{\textit{#1}}}}
\newcommand{\CommentVarTok}[1]{\textcolor[rgb]{0.56,0.35,0.01}{\textbf{\textit{#1}}}}
\newcommand{\OtherTok}[1]{\textcolor[rgb]{0.56,0.35,0.01}{#1}}
\newcommand{\FunctionTok}[1]{\textcolor[rgb]{0.00,0.00,0.00}{#1}}
\newcommand{\VariableTok}[1]{\textcolor[rgb]{0.00,0.00,0.00}{#1}}
\newcommand{\ControlFlowTok}[1]{\textcolor[rgb]{0.13,0.29,0.53}{\textbf{#1}}}
\newcommand{\OperatorTok}[1]{\textcolor[rgb]{0.81,0.36,0.00}{\textbf{#1}}}
\newcommand{\BuiltInTok}[1]{#1}
\newcommand{\ExtensionTok}[1]{#1}
\newcommand{\PreprocessorTok}[1]{\textcolor[rgb]{0.56,0.35,0.01}{\textit{#1}}}
\newcommand{\AttributeTok}[1]{\textcolor[rgb]{0.77,0.63,0.00}{#1}}
\newcommand{\RegionMarkerTok}[1]{#1}
\newcommand{\InformationTok}[1]{\textcolor[rgb]{0.56,0.35,0.01}{\textbf{\textit{#1}}}}
\newcommand{\WarningTok}[1]{\textcolor[rgb]{0.56,0.35,0.01}{\textbf{\textit{#1}}}}
\newcommand{\AlertTok}[1]{\textcolor[rgb]{0.94,0.16,0.16}{#1}}
\newcommand{\ErrorTok}[1]{\textcolor[rgb]{0.64,0.00,0.00}{\textbf{#1}}}
\newcommand{\NormalTok}[1]{#1}
\usepackage{graphicx,grffile}
\makeatletter
\def\maxwidth{\ifdim\Gin@nat@width>\linewidth\linewidth\else\Gin@nat@width\fi}
\def\maxheight{\ifdim\Gin@nat@height>\textheight\textheight\else\Gin@nat@height\fi}
\makeatother
% Scale images if necessary, so that they will not overflow the page
% margins by default, and it is still possible to overwrite the defaults
% using explicit options in \includegraphics[width, height, ...]{}
\setkeys{Gin}{width=\maxwidth,height=\maxheight,keepaspectratio}
\IfFileExists{parskip.sty}{%
\usepackage{parskip}
}{% else
\setlength{\parindent}{0pt}
\setlength{\parskip}{6pt plus 2pt minus 1pt}
}
\setlength{\emergencystretch}{3em}  % prevent overfull lines
\providecommand{\tightlist}{%
  \setlength{\itemsep}{0pt}\setlength{\parskip}{0pt}}
\setcounter{secnumdepth}{0}
% Redefines (sub)paragraphs to behave more like sections
\ifx\paragraph\undefined\else
\let\oldparagraph\paragraph
\renewcommand{\paragraph}[1]{\oldparagraph{#1}\mbox{}}
\fi
\ifx\subparagraph\undefined\else
\let\oldsubparagraph\subparagraph
\renewcommand{\subparagraph}[1]{\oldsubparagraph{#1}\mbox{}}
\fi

%%% Use protect on footnotes to avoid problems with footnotes in titles
\let\rmarkdownfootnote\footnote%
\def\footnote{\protect\rmarkdownfootnote}

%%% Change title format to be more compact
\usepackage{titling}

% Create subtitle command for use in maketitle
\newcommand{\subtitle}[1]{
  \posttitle{
    \begin{center}\large#1\end{center}
    }
}

\setlength{\droptitle}{-2em}

  \title{Drosophila Traumatic Brain Injury}
    \pretitle{\vspace{\droptitle}\centering\huge}
  \posttitle{\par}
    \author{Anji Trujillo - Professor Coon}
    \preauthor{\centering\large\emph}
  \postauthor{\par}
      \predate{\centering\large\emph}
  \postdate{\par}
    \date{December 8, 2018}


\begin{document}
\maketitle

This is an R Markdown Document describing the proteomics data collected
for the Drosophila Traumatic Brain Injury Project in collaboration with
Professor Wassarman and Becky Steinbrech. The samples analyzed were
derived from heads and hemolymph of Drosophilla melanogaster. For the
head samples, there are 17 time points with a control and a traumatic
brain injury (TBI) sample at each time point. The hemolymph sample is a
shorter temporal study with control and traumatic brain injury samples
for 4 time points.

Proteomics data was collected September 22, 2018 and searched on October
16, 2018. The initial analysis was performed December 9, 2018.

The head and hemolymph data was searched separately with an appropriate
fractionated sample to increase our protein group resolution.

\subsection{Load data}\label{load-data}

\begin{verbatim}
## Number of protein groups in Hemolymph Data before any filtering of missing measurements: 9746
\end{verbatim}

\begin{verbatim}
## Number of protein groups in Heads Data before any filtering of missing measurements: 10290
\end{verbatim}

The data analysis was performed using R. The following functions were
written to manipulate the data:\\
1. Remove Contaminants and Reverse Sequences\\
2. Subset columns only with protein ID's, number identifier, and Gene
Name\\
3. 50\% data cut off, filters data to exclude proteins that have more
than 50\% of measurements missing 4. Plotting Standard Deviation of
Proteins 5. Fuzzy c-means preperation for data that has not been imputed
yet\\
6. Fuzzy c-means for data that has been previously imputed (in this case
Bayesian Statistics from PCAMethods package)

\subsection{Hemolymph Data}\label{hemolymph-data}

The hemolymph data includes samples for the first 24 hours (1,4,8,24
hrs) after traumatic brain injury has been implemented. The data
analysis will:\\
1. Clean proteomics data from MaxQuant search algorithm output\\
2. Distribution of proteomic data before and after Bayesian
imputation.\\
3. PCA plots describing the variation in protein groups (scores) and the
variation in the samples (loading) 4. Fuzzy Clustering and hierarchical
plots\\
5. Clustering is also performed on fold change data

Number of missing measurement in Hemo dataset: 31443

Filter and exclude protein groups that are missing over 50\% of
measurements

\begin{verbatim}
## [1] "Number of protein groups that have over 50% missing measurements: 3151"
## [1] "Protein groups that pass the 50% filteration: 6030"
## [1] "Number of protein groups removed from dataset: 3151"
\end{verbatim}

Implement a bayesian pca imputation for data that is log2 transformed.
The R package used is called
\href{https://www.bioconductor.org/packages/devel/bioc/manuals/pcaMethods/man/pcaMethods.pdf}{pcaMethod}.

\begin{Shaded}
\begin{Highlighting}[]
\CommentTok{#PCA function utelizing method=bpca "bayesian"}
\NormalTok{pc_hemolog2 <-}\StringTok{ }\KeywordTok{pca}\NormalTok{(hemo_50percent_log2, }\DataTypeTok{nPcs =} \DecValTok{3}\NormalTok{, }\DataTypeTok{method =} \StringTok{"bpca"}\NormalTok{) }\CommentTok{#pca method}
\CommentTok{#extract imputed data set for hemo data}
\NormalTok{imputed_hemo <-}\StringTok{ }\KeywordTok{completeObs}\NormalTok{(pc_hemolog2)}

\KeywordTok{write.csv}\NormalTok{(imputed_hemo, }\StringTok{"E:/Projects/Proteomics/DorsophilaHead_Experiment/Imputed_hemo_log2.csv"}\NormalTok{)}
\end{Highlighting}
\end{Shaded}

The distribution of the data bfore and after Bayesian imputation,
plotted below.
\includegraphics{Markdown_FuzzyClustering_Drosophila_20181126_files/figure-latex/density_hemo-1.pdf}

\includegraphics{Markdown_FuzzyClustering_Drosophila_20181126_files/figure-latex/density_imputed_hemo-1.pdf}

A global view of the data can visualized by PCA analysis. Imputed data
is used from here on forward\\
* The first PCA is a plot where each dot represents a protein (6,030
proteins), the text denote which sample the protein group derives from,
and the color indicates either control or TBI sample type.\\
* In the second PCA plot the loadings describe the variance in each of
the hemolymph samples determined by protein group. Each dot is a
hymolymph sample and the color indicates either control or TBI sample
type.\\
\includegraphics{Markdown_FuzzyClustering_Drosophila_20181126_files/figure-latex/pca_plot-1.pdf}
\includegraphics{Markdown_FuzzyClustering_Drosophila_20181126_files/figure-latex/pca_plot-2.pdf}

\subsection{Hemo mFuzz Soft
Clustering}\label{hemo-mfuzz-soft-clustering}

Clustering will be implemnted on the proteomics data set that that has
been organized such that control samples are the first and TBI samples
are following.

Prior to fuzzy clustering the variation for each protein plotted.

\begin{verbatim}
## 96 Proteins have a standard deviation greater than  0.1 .
\end{verbatim}

\includegraphics{Markdown_FuzzyClustering_Drosophila_20181126_files/figure-latex/mfuzz_bayesian_imputed-1.pdf}

Soft clustering is implemented in the function mfuzz using a fuzzy
c-means algorithm from
\href{https://cran.r-project.org/web/packages/e1071/e1071.pdf}{e1071}
package.

Fuzzy clustering is one many types of clustering algoriths to choose
from e.g.~K-means, hierarchical, bayesian, etc. I will show fuzzy
clustering along with hierarchical clustering to visualize the protein
trends we see for through out the timepoint. The clustering will be
performed first on data set that includes both the control and TBI
samples and then performed on the fold change data relative to control.

To begin th we must first optimize the parameters for fuzzy clustering
we must calculate the fuzzier value as well as the number of clusters to
divide the data into. The fuzzier ``m'' and the number of clusters ``c''
must be chosen in advanced. This fuzzy type of clustering is
advantageous over hard cluster (e.g.~k-means) which commonly detects
clusters of random data.

The function mestimate

\begin{verbatim}
## [1] "fuzzier m = 1.4410339130003"
\end{verbatim}

\begin{Shaded}
\begin{Highlighting}[]
\CommentTok{# Find the error associated with number of cluster 2-15}
\ControlFlowTok{for}\NormalTok{(i }\ControlFlowTok{in} \DecValTok{2}\OperatorTok{:}\DecValTok{15}\NormalTok{)\{}
\NormalTok{  c1 <-}\StringTok{ }\KeywordTok{mfuzz}\NormalTok{(y, }\DataTypeTok{c=}\NormalTok{i, }\DataTypeTok{m=}\NormalTok{m1)}
\NormalTok{  error <-}\StringTok{ }\KeywordTok{rbind}\NormalTok{(error, }\KeywordTok{c}\NormalTok{(i,c1}\OperatorTok{$}\NormalTok{withinerror))}
\NormalTok{\}}
\end{Highlighting}
\end{Shaded}

Look at the error calculated for each of the clusters.
\includegraphics{Markdown_FuzzyClustering_Drosophila_20181126_files/figure-latex/Error_plot-1.pdf}

Fuzzy clusters are plotted for 6 clusters.\\
\emph{Note first 4 samples are Controls and next set of 4 are TBI
samples\\
}Note a seed has been set to keep clusters consistent.\\
\includegraphics{Markdown_FuzzyClustering_Drosophila_20181126_files/figure-latex/numberOfClusters_beforeMembership-1.pdf}

Filter out proteins from clusters with a membership score of greater
than 70\%. Accounting for the majority of the temporal trend.
\includegraphics{Markdown_FuzzyClustering_Drosophila_20181126_files/figure-latex/numberOfClusters_70percentMembership-1.pdf}

How many proteins have a membership score greater than 70\%, count TRUE.

\begin{verbatim}
## 
## FALSE  TRUE 
##  4111  1919
\end{verbatim}

Heat map of hemo samples that have a membership value greater than 70\%.
Global view of hemo samples.
\includegraphics{Markdown_FuzzyClustering_Drosophila_20181126_files/figure-latex/heatmap_Highmembership-1.pdf}

\subsection{Hemo mFuzz Soft Clustering - fold
change}\label{hemo-mfuzz-soft-clustering---fold-change}

Fuzzy clustering is performed on fold change data set. The fold change
was calculated relative to controls. We will look at clustering of
proteins that have a fold change greater than 1.2. There are fold
changes in the hemolymph data that are greater than 8, and these
proteins with extreme fold changes must be looked at in depth. However
to reduce their bias on clustering, I have implemented a binning
strategy that takes a look at the data set between logfold -2 and 2,
compressing the larger fold changes down to fold change of 2.1. This way
we can see how the protein groups bin at a more granular level.

I have added fuzzy clustering at the end of this section, however I dont
believe fuzzy clustering will be the best strategy for visualizing the
clusters. There are hierarchical clusters and a clusplot at the
beginning of this section, to help determine the number of clusters the
data partitions into.

\begin{Shaded}
\begin{Highlighting}[]
\NormalTok{imputed_hemo_fold <-}\StringTok{ }\NormalTok{imputed_hemo_reorder[,}\DecValTok{5}\OperatorTok{:}\DecValTok{8}\NormalTok{]}\OperatorTok{-}\NormalTok{imputed_hemo_reorder[,}\DecValTok{1}\OperatorTok{:}\DecValTok{4}\NormalTok{]}

\KeywordTok{write.csv}\NormalTok{(imputed_hemo_fold, }\StringTok{"E:/Projects/Proteomics/DorsophilaHead_Experiment/Imputed_hemo_foldchange.csv"}\NormalTok{)}

\CommentTok{# subset features with fold change greater than 1.2 to cluster}
\NormalTok{important_features_hemo_foldchange <-}\StringTok{ }\NormalTok{imputed_hemo_fold[}\KeywordTok{rowSums}\NormalTok{(}\KeywordTok{abs}\NormalTok{(imputed_hemo_fold)}\OperatorTok{>}\FloatTok{1.2}\NormalTok{)}\OperatorTok{>}\DecValTok{0}\NormalTok{,]}

\CommentTok{# bin protein groups so that we describe the changes within fold changes of -2 and 2}
\NormalTok{important_features_foldchange_compress <-}\StringTok{ }\NormalTok{important_features_hemo_foldchange}
\NormalTok{important_features_foldchange_compress[important_features_foldchange_compress }\OperatorTok{<}\StringTok{ }\OperatorTok{-}\DecValTok{2}\NormalTok{] =}\StringTok{ }\OperatorTok{-}\FloatTok{2.1}
\NormalTok{important_features_foldchange_compress[important_features_foldchange_compress }\OperatorTok{>}\StringTok{ }\DecValTok{2}\NormalTok{] =}\StringTok{ }\FloatTok{2.1}

\CommentTok{# Protein groups with fold change greater than 1.2}
\KeywordTok{table}\NormalTok{(}\KeywordTok{rowSums}\NormalTok{(}\KeywordTok{abs}\NormalTok{(imputed_hemo_fold)}\OperatorTok{>}\FloatTok{1.2}\NormalTok{)}\OperatorTok{>}\DecValTok{0}\NormalTok{)}
\end{Highlighting}
\end{Shaded}

\begin{verbatim}
## 
## FALSE  TRUE 
##  4522  1508
\end{verbatim}

\begin{Shaded}
\begin{Highlighting}[]
\CommentTok{# Protein groups with fold change greater than 2}
\KeywordTok{table}\NormalTok{(}\KeywordTok{rowSums}\NormalTok{(}\KeywordTok{abs}\NormalTok{(imputed_hemo_fold)}\OperatorTok{>}\DecValTok{2}\NormalTok{)}\OperatorTok{>}\DecValTok{0}\NormalTok{)}
\end{Highlighting}
\end{Shaded}

\begin{verbatim}
## 
## FALSE  TRUE 
##  5661   369
\end{verbatim}

\begin{Shaded}
\begin{Highlighting}[]
\CommentTok{# Protein groups with fold change greater than 4}
\KeywordTok{table}\NormalTok{(}\KeywordTok{rowSums}\NormalTok{(}\KeywordTok{abs}\NormalTok{(imputed_hemo_fold)}\OperatorTok{>}\DecValTok{4}\NormalTok{)}\OperatorTok{>}\DecValTok{0}\NormalTok{)}
\end{Highlighting}
\end{Shaded}

\begin{verbatim}
## 
## FALSE  TRUE 
##  5984    46
\end{verbatim}

\begin{Shaded}
\begin{Highlighting}[]
\CommentTok{# Protein groups with fold change greater than 8}
\KeywordTok{table}\NormalTok{(}\KeywordTok{rowSums}\NormalTok{(}\KeywordTok{abs}\NormalTok{(imputed_hemo_fold)}\OperatorTok{>}\DecValTok{8}\NormalTok{)}\OperatorTok{>}\DecValTok{0}\NormalTok{)}
\end{Highlighting}
\end{Shaded}

\begin{verbatim}
## 
## FALSE  TRUE 
##  6027     3
\end{verbatim}

\includegraphics{Markdown_FuzzyClustering_Drosophila_20181126_files/figure-latex/heatmap_important_features_1.2-1.pdf}
\includegraphics{Markdown_FuzzyClustering_Drosophila_20181126_files/figure-latex/heatmap_important_features_1.2-2.pdf}

\begin{Shaded}
\begin{Highlighting}[]
\NormalTok{fold_hclust <-}\StringTok{ }\KeywordTok{hclust}\NormalTok{(}\KeywordTok{dist}\NormalTok{(important_features_foldchange_compress, }\DataTypeTok{method =} \StringTok{"euclidean"}\NormalTok{))}
\NormalTok{clusters <-}\StringTok{ }\KeywordTok{cutree}\NormalTok{(fold_hclust, }\DataTypeTok{k =} \DecValTok{6}\NormalTok{) }\CommentTok{#k = 20)}

\KeywordTok{clusplot}\NormalTok{(important_features_foldchange_compress, clusters, }\DataTypeTok{lines =} \DecValTok{0}\NormalTok{)}
\end{Highlighting}
\end{Shaded}

\includegraphics{Markdown_FuzzyClustering_Drosophila_20181126_files/figure-latex/mfuzz_clustering-1.pdf}

\begin{Shaded}
\begin{Highlighting}[]
\KeywordTok{plot}\NormalTok{(fold_hclust, }\DataTypeTok{label=} \OtherTok{FALSE}\NormalTok{)}
\KeywordTok{rect.hclust}\NormalTok{(fold_hclust, }\DataTypeTok{k=}\DecValTok{6}\NormalTok{, }\DataTypeTok{border =} \StringTok{"red"}\NormalTok{)}
\end{Highlighting}
\end{Shaded}

\includegraphics{Markdown_FuzzyClustering_Drosophila_20181126_files/figure-latex/mfuzz_clustering-2.pdf}

\section{Fold Change Hemo Data Fuzzy
Clustering}\label{fold-change-hemo-data-fuzzy-clustering}

\begin{Shaded}
\begin{Highlighting}[]
\CommentTok{#start with the re-ordered hemo data}
\NormalTok{z =}\StringTok{ }\NormalTok{important_features_foldchange_compress }\CommentTok{# change accordingly to use in mfuzzy setup funtions}

\NormalTok{exprValues.s_bayesian <-}\StringTok{ }\KeywordTok{fuzzyprep_usepreviousImputation_foldchange}\NormalTok{(z)}
\end{Highlighting}
\end{Shaded}

\begin{verbatim}
## 1 Proteins have a standard deviation greater than  0.1 .
\end{verbatim}

\includegraphics{Markdown_FuzzyClustering_Drosophila_20181126_files/figure-latex/mfuzz_bayesian_imputed_foldchange-1.pdf}

\begin{Shaded}
\begin{Highlighting}[]
\NormalTok{error <-}\StringTok{ }\OtherTok{NA} \CommentTok{#set error to NA}

\NormalTok{y =}\StringTok{ }\NormalTok{exprValues.s_bayesian }\CommentTok{#change according to version of data to perform fuzzy clustering}
\end{Highlighting}
\end{Shaded}

The function mestimate

\begin{Shaded}
\begin{Highlighting}[]
\NormalTok{m1 <-}\StringTok{ }\KeywordTok{mestimate}\NormalTok{(y) }

\CommentTok{# }
\ControlFlowTok{for}\NormalTok{(i }\ControlFlowTok{in} \DecValTok{2}\OperatorTok{:}\DecValTok{15}\NormalTok{)\{}
\NormalTok{  c1 <-}\StringTok{ }\KeywordTok{mfuzz}\NormalTok{(y, }\DataTypeTok{c=}\NormalTok{i, }\DataTypeTok{m=}\NormalTok{m1)}
\NormalTok{  error <-}\StringTok{ }\KeywordTok{rbind}\NormalTok{(error, }\KeywordTok{c}\NormalTok{(i,c1}\OperatorTok{$}\NormalTok{withinerror))}
\NormalTok{\}}
\end{Highlighting}
\end{Shaded}

Plot error calculated from imputed hemo data set that has been
reordered.
\includegraphics{Markdown_FuzzyClustering_Drosophila_20181126_files/figure-latex/error_hemo_foldChange-1.pdf}

\includegraphics{Markdown_FuzzyClustering_Drosophila_20181126_files/figure-latex/numberOfClusters_foldchange-1.pdf}
\includegraphics{Markdown_FuzzyClustering_Drosophila_20181126_files/figure-latex/numberOfClusters_foldchange-2.pdf}

\subsection{Head Data}\label{head-data}

\begin{Shaded}
\begin{Highlighting}[]
\NormalTok{proteinGroups_dros_heads <-}\StringTok{ }\KeywordTok{RemoveContaminants}\NormalTok{(proteinGroups_dros_heads)}
\KeywordTok{print}\NormalTok{(}\KeywordTok{dim}\NormalTok{(proteinGroups_dros_heads))}
\end{Highlighting}
\end{Shaded}

\begin{verbatim}
## [1] 9365  335
\end{verbatim}

\begin{Shaded}
\begin{Highlighting}[]
\NormalTok{proteinGroups_dros_heads <-}\StringTok{ }\KeywordTok{subsetLFQ}\NormalTok{(proteinGroups_dros_heads) }
\KeywordTok{print}\NormalTok{(}\KeywordTok{dim}\NormalTok{(proteinGroups_dros_heads))}
\end{Highlighting}
\end{Shaded}

\begin{verbatim}
## [1] 9365   38
\end{verbatim}

Number of missing measurement in Head dataset: 83521

Filter and exclude protein groups that are missing over 50\% of
measurements

\begin{verbatim}
## [1] "Number of protein groups that have over 50% missing measurements: 2190"
## [1] "Protein groups that pass the 50% filteration: 7175"
## [1] "Number of protein groups removed from dataset: 2190"
\end{verbatim}

Implement a bayesian pca imputation for data that is log2 transformed.
The R package used is called
\href{https://www.bioconductor.org/packages/devel/bioc/manuals/pcaMethods/man/pcaMethods.pdf}{pcaMethod}.

\begin{Shaded}
\begin{Highlighting}[]
\CommentTok{#subset matrix containing only samples that have been filtered 50% threshold}
\NormalTok{head_50percent <-}\StringTok{ }\KeywordTok{as.matrix}\NormalTok{(filtered_dros_head_50percent[,}\KeywordTok{c}\NormalTok{(}\DecValTok{5}\OperatorTok{:}\DecValTok{38}\NormalTok{)])}
\CommentTok{#log2 transform}
\NormalTok{head_50percent_log2 <-}\StringTok{ }\KeywordTok{log2}\NormalTok{(head_50percent)}
\CommentTok{#set row names to match protein group identifer number}
\KeywordTok{rownames}\NormalTok{(head_50percent_log2) <-}\StringTok{ }\NormalTok{filtered_dros_head_50percent}\OperatorTok{$}\NormalTok{id}

\KeywordTok{colnames}\NormalTok{(head_50percent_log2) <-}\StringTok{ }\KeywordTok{as.numeric}\NormalTok{(}\KeywordTok{gsub}\NormalTok{(}\StringTok{".*_"}\NormalTok{,}\StringTok{""}\NormalTok{,}\KeywordTok{colnames}\NormalTok{(head_50percent_log2)))}

\NormalTok{col_levels <-}\StringTok{ }\KeywordTok{levels}\NormalTok{(}\KeywordTok{colnames}\NormalTok{(head_50percent_log2)) <-}\StringTok{ }\KeywordTok{seq}\NormalTok{(}\DecValTok{1}\OperatorTok{:}\DecValTok{34}\NormalTok{)}
\KeywordTok{colnames}\NormalTok{(head_50percent_log2) <-}\StringTok{ }\KeywordTok{order}\NormalTok{(}\KeywordTok{colnames}\NormalTok{(head_50percent_log2))}

\NormalTok{col_names <-}\StringTok{ }\KeywordTok{names}\NormalTok{(}\KeywordTok{colnames}\NormalTok{(head_50percent_log2)) <-}\StringTok{ }\KeywordTok{order}\NormalTok{(}\KeywordTok{colnames}\NormalTok{(head_50percent_log2))}

\NormalTok{head_meta <-}\StringTok{ }\KeywordTok{data.frame}\NormalTok{(}\DataTypeTok{HeadTimePoints =} \KeywordTok{order}\NormalTok{(col_levels),}\DataTypeTok{Sample_Type=}\KeywordTok{rep}\NormalTok{(}\KeywordTok{c}\NormalTok{(}\StringTok{"Control"}\NormalTok{,}\StringTok{"TBI"}\NormalTok{),}\DecValTok{17}\NormalTok{))}

\NormalTok{head_50percent_log2 <-}\StringTok{ }\NormalTok{head_50percent_log2[,}\KeywordTok{names}\NormalTok{(}\KeywordTok{colnames}\NormalTok{(head_50percent_log2))]}
\end{Highlighting}
\end{Shaded}

\begin{Shaded}
\begin{Highlighting}[]
\CommentTok{#PCA function utelizing method=bpca "bayesian"}
\NormalTok{pc_headlog2 <-}\StringTok{ }\KeywordTok{pca}\NormalTok{(head_50percent_log2, }\DataTypeTok{nPcs =} \DecValTok{3}\NormalTok{, }\DataTypeTok{method =} \StringTok{"bpca"}\NormalTok{) }\CommentTok{#pca method}
\CommentTok{#extract imputed data set for hemo data}
\NormalTok{imputed_head <-}\StringTok{ }\KeywordTok{completeObs}\NormalTok{(pc_headlog2)}

\KeywordTok{write.csv}\NormalTok{(imputed_head, }\StringTok{"E:/Projects/Proteomics/DorsophilaHead_Experiment/Imputed_head.csv"}\NormalTok{)}
\end{Highlighting}
\end{Shaded}

\includegraphics{Markdown_FuzzyClustering_Drosophila_20181126_files/figure-latex/pca_plot_head-1.pdf}
\includegraphics{Markdown_FuzzyClustering_Drosophila_20181126_files/figure-latex/pca_plot_head-2.pdf}

The distribution of the data is plotted below before and after
imputation
\includegraphics{Markdown_FuzzyClustering_Drosophila_20181126_files/figure-latex/density_head-1.pdf}

\includegraphics{Markdown_FuzzyClustering_Drosophila_20181126_files/figure-latex/density_imputed_head-1.pdf}

\subsection{Head mFuzz Soft
Clustering}\label{head-mfuzz-soft-clustering}

Clustering will be implemnted on data set that contains control samples
first 17 and TBI samples remaining 17.

First we will look at the variation of the data by calculating the
standard deviation of each protein.

\begin{verbatim}
## 105 Proteins have a standard deviation greater than  0.1 .
\end{verbatim}

\includegraphics{Markdown_FuzzyClustering_Drosophila_20181126_files/figure-latex/mfuzz_bayesian_headimputed-1.pdf}

Soft clustering is implemented in the function mfuzz using a fuzzy
c-means algorithm from e1071 package.

To optimize the parameters for fuzzy clustering we must calculate the
fuzzier value as well as the number of clusters to divide the data into.
The fuzzier ``m'' and the number of clusters ``c'' must be chosen in
advanced. For ``m'' we choose a value that prevents clustering of random
data. This fuzzy type of clustering is advantageous over hard cluster
(e.g.~k-means) which commonly detects clusters of random data.

The function mestimate

\begin{verbatim}
## [1] "fuzzier m = 1.065266572006"
\end{verbatim}

\begin{Shaded}
\begin{Highlighting}[]
\CommentTok{# Find the error associated with number of cluster 2-15}
\ControlFlowTok{for}\NormalTok{(i }\ControlFlowTok{in} \DecValTok{2}\OperatorTok{:}\DecValTok{20}\NormalTok{)\{}
\NormalTok{  c1 <-}\StringTok{ }\KeywordTok{mfuzz}\NormalTok{(y, }\DataTypeTok{c=}\NormalTok{i, }\DataTypeTok{m=}\NormalTok{m1)}
\NormalTok{  error <-}\StringTok{ }\KeywordTok{rbind}\NormalTok{(error, }\KeywordTok{c}\NormalTok{(i,c1}\OperatorTok{$}\NormalTok{withinerror))}
\NormalTok{\}}
\end{Highlighting}
\end{Shaded}

Plot error calculated from the number of clusters used.
\includegraphics{Markdown_FuzzyClustering_Drosophila_20181126_files/figure-latex/Error_plot_head-1.pdf}

Fuzzy clusters are plotted for 6 clusters.\\
\emph{Note first 17 samples are Controls and next set of 17 are TBI
samples\\
}Note a seed has been set to keep clusters consistent.\\
\includegraphics{Markdown_FuzzyClustering_Drosophila_20181126_files/figure-latex/numberOfClusters_beforeMembership_head-1.pdf}
\includegraphics{Markdown_FuzzyClustering_Drosophila_20181126_files/figure-latex/numberOfClusters_beforeMembership_head-2.pdf}

How many proteins have a membership score greater than 70\%, count TRUE.

\begin{verbatim}
## 
## FALSE  TRUE 
##  2023  5152
\end{verbatim}

Heat map of head samples that have a membership value greater than 70\%.
Global view of head samples.
\includegraphics{Markdown_FuzzyClustering_Drosophila_20181126_files/figure-latex/heatmap_Highmembership_head-1.pdf}

\subsection{Head mFuzz Soft Clustering - fold
change}\label{head-mfuzz-soft-clustering---fold-change}

\begin{Shaded}
\begin{Highlighting}[]
\NormalTok{imputed_head_fold <-}\StringTok{ }\NormalTok{imputed_head_reorder[,}\DecValTok{18}\OperatorTok{:}\DecValTok{34}\NormalTok{]}\OperatorTok{-}\NormalTok{imputed_head_reorder[,}\DecValTok{1}\OperatorTok{:}\DecValTok{17}\NormalTok{]}

\KeywordTok{write.csv}\NormalTok{(imputed_head_fold, }\StringTok{"E:/Projects/Proteomics/DorsophilaHead_Experiment/Imputed_head_foldchange.csv"}\NormalTok{)}

\CommentTok{# subset features with fold change greater than 1.2 to cluster}
\NormalTok{important_features_head_foldchange <-}\StringTok{ }\NormalTok{imputed_head_fold[}\KeywordTok{rowSums}\NormalTok{(}\KeywordTok{abs}\NormalTok{(imputed_head_fold)}\OperatorTok{>}\FloatTok{1.2}\NormalTok{)}\OperatorTok{>}\DecValTok{0}\NormalTok{,]}


\KeywordTok{table}\NormalTok{(}\KeywordTok{rowSums}\NormalTok{(}\KeywordTok{abs}\NormalTok{(imputed_head_fold)}\OperatorTok{>}\FloatTok{1.2}\NormalTok{)}\OperatorTok{>}\DecValTok{0}\NormalTok{)}
\end{Highlighting}
\end{Shaded}

\begin{verbatim}
## 
## FALSE  TRUE 
##  4807  2368
\end{verbatim}

\begin{Shaded}
\begin{Highlighting}[]
\KeywordTok{table}\NormalTok{(}\KeywordTok{rowSums}\NormalTok{(}\KeywordTok{abs}\NormalTok{(imputed_head_fold)}\OperatorTok{>}\DecValTok{2}\NormalTok{)}\OperatorTok{>}\DecValTok{0}\NormalTok{)}
\end{Highlighting}
\end{Shaded}

\begin{verbatim}
## 
## FALSE  TRUE 
##  6196   979
\end{verbatim}

\begin{Shaded}
\begin{Highlighting}[]
\KeywordTok{table}\NormalTok{(}\KeywordTok{rowSums}\NormalTok{(}\KeywordTok{abs}\NormalTok{(imputed_head_fold)}\OperatorTok{>}\DecValTok{4}\NormalTok{)}\OperatorTok{>}\DecValTok{0}\NormalTok{)}
\end{Highlighting}
\end{Shaded}

\begin{verbatim}
## 
## FALSE  TRUE 
##  6958   217
\end{verbatim}

\begin{Shaded}
\begin{Highlighting}[]
\KeywordTok{table}\NormalTok{(}\KeywordTok{rowSums}\NormalTok{(}\KeywordTok{abs}\NormalTok{(imputed_head_fold)}\OperatorTok{>}\DecValTok{8}\NormalTok{)}\OperatorTok{>}\DecValTok{0}\NormalTok{)}
\end{Highlighting}
\end{Shaded}

\begin{verbatim}
## 
## FALSE  TRUE 
##  7151    24
\end{verbatim}

\begin{Shaded}
\begin{Highlighting}[]
\CommentTok{# bin protein groups so that we describe the changes within fold changes of -2 and 2}
\NormalTok{important_features_foldchange_compress <-}\StringTok{ }\NormalTok{important_features_head_foldchange}
\NormalTok{important_features_foldchange_compress[important_features_foldchange_compress }\OperatorTok{<}\StringTok{ }\OperatorTok{-}\DecValTok{2}\NormalTok{] =}\StringTok{ }\OperatorTok{-}\FloatTok{2.1}
\NormalTok{important_features_foldchange_compress[important_features_foldchange_compress }\OperatorTok{>}\StringTok{ }\DecValTok{2}\NormalTok{] =}\StringTok{ }\FloatTok{2.1}
\end{Highlighting}
\end{Shaded}

\includegraphics{Markdown_FuzzyClustering_Drosophila_20181126_files/figure-latex/heatmap_important_features_1.2_head-1.pdf}
\includegraphics{Markdown_FuzzyClustering_Drosophila_20181126_files/figure-latex/heatmap_important_features_1.2_head-2.pdf}

\begin{Shaded}
\begin{Highlighting}[]
\NormalTok{fold_hclust <-}\StringTok{ }\KeywordTok{hclust}\NormalTok{(}\KeywordTok{dist}\NormalTok{(important_features_foldchange_compress, }\DataTypeTok{method =} \StringTok{"euclidean"}\NormalTok{))}
\NormalTok{clusters <-}\StringTok{ }\KeywordTok{cutree}\NormalTok{(fold_hclust, }\DataTypeTok{k =} \DecValTok{6}\NormalTok{) }\CommentTok{#k = 20)}

\KeywordTok{clusplot}\NormalTok{(important_features_foldchange_compress, clusters, }\DataTypeTok{lines =} \DecValTok{0}\NormalTok{)}
\end{Highlighting}
\end{Shaded}

\includegraphics{Markdown_FuzzyClustering_Drosophila_20181126_files/figure-latex/mfuzz_clustering_head-1.pdf}

\begin{Shaded}
\begin{Highlighting}[]
\KeywordTok{plot}\NormalTok{(fold_hclust, }\DataTypeTok{label=} \OtherTok{FALSE}\NormalTok{)}
\KeywordTok{rect.hclust}\NormalTok{(fold_hclust, }\DataTypeTok{k=}\DecValTok{6}\NormalTok{, }\DataTypeTok{border =} \StringTok{"red"}\NormalTok{)}
\end{Highlighting}
\end{Shaded}

\includegraphics{Markdown_FuzzyClustering_Drosophila_20181126_files/figure-latex/mfuzz_clustering_head-2.pdf}

\section{Fold Change Head Data Fuzzy
Clustering}\label{fold-change-head-data-fuzzy-clustering}

\begin{Shaded}
\begin{Highlighting}[]
\CommentTok{#start with the re-ordered hemo data}
\NormalTok{z =}\StringTok{ }\NormalTok{important_features_foldchange_compress }\CommentTok{# change accordingly to use in mfuzzy setup funtions}

\NormalTok{exprValues.s_bayesian <-}\StringTok{ }\KeywordTok{fuzzyprep_usepreviousImputation_foldchange}\NormalTok{(z)}
\end{Highlighting}
\end{Shaded}

\begin{verbatim}
## 0 Proteins have a standard deviation greater than  0.1 .
\end{verbatim}

\includegraphics{Markdown_FuzzyClustering_Drosophila_20181126_files/figure-latex/mfuzz_bayesian_imputed_foldchange_head-1.pdf}

\begin{Shaded}
\begin{Highlighting}[]
\NormalTok{error <-}\StringTok{ }\OtherTok{NA} \CommentTok{#set error to NA}

\NormalTok{y =}\StringTok{ }\NormalTok{exprValues.s_bayesian }\CommentTok{#change according to version of data to perform fuzzy clustering}
\end{Highlighting}
\end{Shaded}

The function mestimate

\begin{Shaded}
\begin{Highlighting}[]
\NormalTok{m1 <-}\StringTok{ }\KeywordTok{mestimate}\NormalTok{(y) }

\CommentTok{# }
\ControlFlowTok{for}\NormalTok{(i }\ControlFlowTok{in} \DecValTok{2}\OperatorTok{:}\DecValTok{20}\NormalTok{)\{}
\NormalTok{  c1 <-}\StringTok{ }\KeywordTok{mfuzz}\NormalTok{(y, }\DataTypeTok{c=}\NormalTok{i, }\DataTypeTok{m=}\NormalTok{m1)}
\NormalTok{  error <-}\StringTok{ }\KeywordTok{rbind}\NormalTok{(error, }\KeywordTok{c}\NormalTok{(i,c1}\OperatorTok{$}\NormalTok{withinerror))}
\NormalTok{\}}
\end{Highlighting}
\end{Shaded}

Plot error calculated from imputed hemo data set that has been
reordered.
\includegraphics{Markdown_FuzzyClustering_Drosophila_20181126_files/figure-latex/error_head_foldChange-1.pdf}

\includegraphics{Markdown_FuzzyClustering_Drosophila_20181126_files/figure-latex/numberOfClusters_foldchange_head-1.pdf}
\includegraphics{Markdown_FuzzyClustering_Drosophila_20181126_files/figure-latex/numberOfClusters_foldchange_head-2.pdf}
\includegraphics{Markdown_FuzzyClustering_Drosophila_20181126_files/figure-latex/numberOfClusters_foldchange_head-3.pdf}
\includegraphics{Markdown_FuzzyClustering_Drosophila_20181126_files/figure-latex/numberOfClusters_foldchange_head-4.pdf}

Note that the \texttt{echo\ =\ FALSE} parameter was added to the code
chunk to prevent printing of the R code that generated the plot.


\end{document}
